
    \begin{abstract_online}{P7 - Tortuosidade Arterial da Retina Familiar: Caso Clínico}{%
        \underline{V. Sousa}$^{1}$, L. Martins$^{1}$, S. Penas$^{1}$}{%
        }{%
        $^1$ ULS S. João\newline{}}
        A tortuosidade arteriolar retiniana familiar (FRAT) é uma doença hereditária rara marcada pela tortuosidade progressiva de artérias de segunda e terceira ordem nas regiões macular e peripapilar, que surge normalmente na infância ou no início da idade adulta. Esta condição pode levar a hemorragias intra ou pré-retinianas espontâneas ou relacionadas com o esforço, que podem não apresentar sintomas, excepto quando afetam a mácula. Na maioria dos casos, não existem outros problemas vasculares ou doenças sistémicas associadas, mas, ocasionalmente, as hemorragias retinianas podem preceder o desenvolvimento visível da tortuosidade dos vasos. Neste caso clínico, foram detectados 3 pacientes, familiares diretos com FRAT em diferentes graus da doença. Estes achados diferentes entre eles podem remontar à evolução da patologia possivelmente relacionado com a idade. Foram realizados exames de OCT e retinografia e comparados os achados e confirmado o diagnóstico pelo teste genético.  
    \end{abstract_online}
    