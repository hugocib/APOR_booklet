
    \begin{abstract_online}{P8 - Da imagem à percepção: utilização da inteligência artificial na identificação de padrões ocultos em doentes com glaucoma}{%
        \underline{I. Shcherbinin}$^{1}$, C. Mateus$^{2}$, I. Marques$^{3}$}{%
        }{%
        $^1$ Faculdade de Medicina da Universidade de Coimbra\newline{}$^2$ Escola Superior de Saúde - Instituto Politécnico do Porto\newline{}$^3$ AIBILI}
        Serão analisadas retinografias e historial clínico de um total de 300 participantes, divididos entre um grupo de 150 doentes com glaucoma e um grupo controlo de 150 indivíduos sem patologia ocular diagnosticada. Os doentes com glaucoma serão subdivididos em três grupos: grupo com hipertensão ocular (n=50), suspeitas de glaucoma (n=50) e doentes diagnosticados com glaucoma primário de ângulo aberto (n=50). Serão coletados dados clínicos (pressão intra-ocular; cup-to-disc ratio), historial médico, exames oftalmológicos relevantes para o diagnóstico (pec; oct; retinografia). Serão utilizados algoritmos de machine learning para extrair padrões relevantes da retinografia e dados clínicos. 
As respetivas retinografias irão passar por um processamento de imagem com ajuste de contraste, resolução e saturação de forma a otimizar o desempenho do algoritmo e garantir uma base de dados uniforme. Algoritmo de deep learning (yolo – you only live once), será treinado com base num labeling detalhado através do software “roboflow” usando os dados para criar modelos de diagnóstico e previsão. Os dados extraídos serão analisados estatisticamente para identificar padrões, associações e diferenças entre os subgrupos de glaucoma e controlo. Testes estatísticos apropriados serão usados para avaliar a relação entre as características e a progressão/risco de glaucoma. Os modelos desenvolvidos serão validados e resultados serão interpretados em relação aos objetivos do estudo e à literatura existente, visando contribuir para a compreensão do diagnóstico e gestão da patologia de forma mais personalizada com recurso a algoritmos de inteligência artificial.
    \end{abstract_online}
    