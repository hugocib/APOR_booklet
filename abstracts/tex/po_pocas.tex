
    \begin{abstract_online}{Paralisia parcial do III par craniano  no contexto de aneurisma intracraneano  - relato de caso }{%
        \underline{I. Poças}$^{1}$, P. Lino$^{2}$}{%
        }{%
        $^1$ Universidade Lusófona de Humanidades e Tecnologias Centro de Estudos Interdisciplinares em Educação e Desenvolvimento: Lisboa\newline{}$^2$ ESTesL-IPL; Hospitais e Clínicas CUF; CHULN; ENSP/NMS-UNL}
        Introdução: As queixas de diplopia, podem surgir de uma série de doenças que variam de benignas a potencialmente fatais, como por exemplo aneurismas. A abordagem diagnóstica inicial para diplopia consiste na anamnese e exame neurológico, oftalmológico detalhado, com foco especial na motilidade ocular; assim como exame imagiológico dirigido. 
        Objetivo: Descrever o caso de uma paralisia parcial do III Par direito, envolvendo a musculatura extrínseca do olho direito, no contexto de um aneurisma cerebral. 
        Metodologia:  Relato de caso de uma paciente do sexo feminino de 60 anos de idade, com diplopia vertical devido a uma paralisia do RS do OD no contexto de um aneurisma cerebral. A avaliação diagnóstica e de seguimento incluiu o exame oftalmológico, ortóptico e neurológico detalhados, assim como exame imagiológico dirigido. Resultados: Paciente recorre a consulta de oftalmologia por diplopia binocular vertical e fotofobia. Ao exame oftalmológico: ptose à direita, com limitação da elevação. O exame ortóptico releva uma Hipertropia do OE com limitação da elevação do OD  com diplopia na dextrosupraversão e supraversão. Foram efetuados exames imagiológicos dirigidos para esclarecimento etiológico que revelaram a existência de uma aneurisma cerebral. 
        Dada a patologia de base, que exigiu intervenção cirurgica; a paciente foi enviada para o departamento de neurocirurgia para tratamento  cirúrgico do aneurisma. Apos resolução  cirúrgica do aneurisma, persistiram défices neurológicos graves como diplopia binocular vertical e  ptose. 
        Conclusão: Quando os exames imagiológicos mostram doenças de base que expliquem a diplopia,  o primeiro passo terapêutico é o tratamento da doença de base, e só após tratamento da mesma é que se explora a compensação oftalmológica da diplopia nomeadamente do ponto de vista cirúrgico. Na fase aguda opta-se, apenas,  pelo tratamento sintomático da diplopia com oclusão alternada ou penalização do olho não dominante com filtros de Banguerter ou  com a utilização de prismas.
    \end{abstract_online}
    