
    \begin{abstract_online}{P5 - Degenerescência macular da idade: A expressão de Metiltransferases é um potencial biomarcador?}{%
        \underline{P. Camacho}$^{1,3}$, E. Ribeiro$^{1}$, B. Pereira$^{2,3}$, C. Ginete$^{1}$, J. Nascimento$^{2}$, S. Barrão$^{4}$, J. Henriques$^{2}$, P. Rosa$^{2,4}$, C. Raposo$^{4}$, A. Almeida$^{4}$, R. Fernandes$^{4}$, S. Sadio$^{4}$, C. Silva$^{1}$, M. Brito$^{1}$}{%
        }{%
        $^1$ H\&TRC-Health \& Technology Research Center, ESTeSL Escola Superior de Tecnologia da Saúde, Instituto Politécnico de Lisboa\newline{}$^2$ Instituto de Retina de Lisboa, IRL, Lisboa\newline{}$^3$  iNOVA4Health, NOVA Medical School, Faculdade de Ciências Médicas, NMS, FCM, Universidade NOVA de Lisboa\newline{}$^4$ Instituto de Oftalmologia Dr. Gama Pinto, IOGP, Lisboa}
        Apesar dos importantes avanços no tratamento da degenerescência macular da idade (DMI) neovascular (nDMI), a falta de opções terapêuticas para as formas não avançadas (cerca de 90\%) e para as formas atróficas avançadas realça a necessidade de identificar potenciais biomarcadores como estratégia de saúde pública. A abordagem multimodal, com o apoio da histologia, tem sido importante na procura de potenciais biomarcadores de AMD. Também a genética tem sido importante, no entanto apenas 40–60\% da doença é explicada por fatores genéticos e a associação com biomarcadores de imagem não invasivos é limitada.\newline{}
Objetivos: Com a crescente importância da epigenética, o estudo MetAllAMD teve como objetivo caracterizar a expressão génica de modeladores epigenéticos (DNMT1, DNMT2A, DNMT3B) em todos os estádios da DMI e estudar a correlação com os novos biomarcadores de imagem.\newline{}
Material e Métodos: Um total de 14 doentes com DMI, com idades compreendidas entre os 59 e os 90 anos, foram incluídos prospectivamente neste estudo. Os participantes foram classificados em DMI precoce/intermédia (iDMI) e DMI avançada (aDMI). Apenas os casos com um exame oftalmológico completo, fundo de olho digital 133º a cores e avaliações SD-OCT foram incluídos. Através de PCR quantitativo em tempo real (qRT-PCR) procedeu-se à quantificação da expressão de genes epigenéticos (DNMT1, DNMT2A, DNMT3B) a partir do RNA total.\newline{}
Resultados: Os participantes com aDMI apresentaram uma acuidade visual inferior (p=0,001) ao grupo com iDMI embora sem alterações significativas ao nível da espessura central da retina nem da espessura mínima da fóvea. Na quantificação da expressão de modeladores epigenéticos verificou-se uma diminuição da DNMT1 (p=0,003), DNMT3A (p=0,004) e DNMT3B (p=0,004) no grupo aDMI.\newline{}
Conclusões: A expressão de genes modeladores epigenéticos (DNMT1, DNMT3A e DNMT3B) surgem alteradas nas diferentes fases da DMI podendo constituir um potencial biomarcador a estudar numa das principais causas de cegueira.
\end{abstract_online}
    