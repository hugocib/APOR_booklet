
    \begin{abstract_online}{P6 - Adaptação de Lente de Contato Escleral na Síndrome de Sjögren }{%
        \underline{R. Nogueira}$^{1}$}{%
        }{%
        $^1$ Optiforum\newline{}}
        Paciente diagnosticada com Síndrome de Sjögren há oito anos, doença autoimune que afeta as glândulas exócrinas e resulta em sintomas de olho seco. Pelo desconforto ocular causado, é necessário introduzir terapia farmacológica para controlar os sintomas, o que nem sempre é aconselhado devido aos efeitos colaterais. Portanto, a adaptação de lentes de contato esclerais surge como uma alternativa para melhorar o conforto ocular e protelar a introdução destes fármacos.
        Na observação inicial, verificou-se uma acuidade visual correspondente a 100\%, constatou-se a presença de sinais de olho seco severo (pontuação 77.50 OSDI e 4 mm Teste de Schirmer), com queratite ponteada extensa e pestanejo incompleto. Pelos sinais observados, optou-se por ensaiar uma lente de contacto neste olho.
        Após a realização de topografia corneana, observação do segmento anterior e através de cálculos para estimativa da separação desejada, testou-se uma lente de contacto prolata de apoio escleral com 17mm e realizou-se o ensaio.
        Após 15 dias de utilização da lente, a paciente reportou excelente conforto e boa visão, com acuidade visual de 100\%. A lente aparentou uma altura sagital regular desde o centro até à zona limbar sem compressão de vasos, constatando-se uma boa adaptação. Na avaliação da superfície ocular, verificou-se a inexistência da queratite extensa previamente verificada. Obteve-se uma melhoria no resultado do teste de Schirmer e redução da pontuação do questionário OSDI, correspondente a Doença de Olho Seco Moderado.
        A adaptação da lente de contato escleral, foi um grande benefício para a qualidade de vida da paciente, uma vez que resultou numa melhoria da Doença de Olho Seco, reduzindo a gravidade de severa para moderada. Por isso, não foi necessário incorporar novos fármacos, que atuem ao nível da melhoria do conforto ocular. Como a adaptação foi um sucesso, tornou-se pertinente o ensaio de uma lente semelhante no olho contralateral.
    \end{abstract_online}
    