
    \begin{abstract_online}{P2 - OCT-A Método de Diagnóstico Não Invasivo nas Doenças da Retina}{%
        \underline{J. Azevedo}$^{1}$, B. Neves$^{1}$, R. Leitão$^{1}$}{%
        }{%
        $^1$ ULS Entre Douro e Vouga\newline{}}
        O crescente aumento da esperança média de vida, a par com a idade ativa da população, acarreta novos desafios ao setor da saúde. É fundamental que os cuidados primários de saúde consigam acompanhar esta evolução, seja com a criação de programas de rastreios, seja com um maior aproveitamento de recursos humanos e monetários, conduzindo a um menor espaço temporal no atendimento ao utente. Exames complementares de diagnósticos inovadores, que exigem menor tempo de execução, uma maior reprodutibilidade, uma boa fiabilidade, são cada vez mais importantes na prática clínica. Neste trabalho analisamos a importância do Tomografia de Coerência Ótica - Angiografia (OCT-A), no serviço de Oftalmologia, e a forma como este permite, de forma não invasiva, o diagnóstico e possibilidade de follow up entre consultas, assim como a avaliação da eficácia do tratamento. Foi escolhido um caso clínico, de um doente com Degenerescência Macular da Idade (DMI), onde em detrimento do método de diagnóstico gold standard, Angiografia Fluoresceínica, exame invasivo, foi realizado um exame OCT-A, combinado com  Tomografia de Coerência Ótica de Domínio Espectral (SD-OCT), para a pesquisas de achados clínicos relevantes como membrana neovascular (MNV). O exame OCT-A isolado demonstra uma alta sensibilidade e especificidade, parâmetros que se mantêm quando associado ao SD-OCT, para diagnóstico e follow up de membrana neovascular (MNV), na DMI.
    \end{abstract_online}
    