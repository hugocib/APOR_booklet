
    \begin{abstract_online}{P3 - Caso Clínico: Switch Terapêutico no Edema Macular Diabético}{%
        \underline{M. Barreto}$^{1}$, C. Santos$^{1}$, I. Silva$^{1}$, S. Tavares$^{1}$}{%
        }{%
        $^1$ ULS Almada \newline{}}
        Introdução: O Edema Macular Diabético é uma das principais causas de perda de visão em idade adulta, sendo uma consequência da Retinopatia Diabética é caracterizada por um aumento da permeabilidade vascular e lesão dos capilares retinianos. A injeção intravítrea é a abordagem terapêutica mais frequentemente utilizada nesta patologia e o seu mecanismo de ação baseia-se no bloqueio do factor de crescimento endotelial vascular e consequente redução da permeabilidade vascular. Para diagnosticar a doença e avaliar o efeito do fármaco realiza-se a Tomografia de Coerência Óptica e a Retinografia, ambos os exames realizados por Ortoptistas. Com a análise deste caso clínico pretende-se verificar a eficácia do switch terapêutico em doentes tratados ad initio com Bevacizumab.\newline{}
        Descrição do caso clínico: Paciente do sexo feminino na casa dos 50 anos e Diabetes Mellitus tipo 2 há 12 anos, sob insulinoterapia e mau controlo metabólico. Recorreu ao serviço de urgência de oftalmologia devido a baixa de visão progressiva, com uma acuidade visual <5/200 bilateral. Iniciou tratamentos com injeções intravítreas de Bevacizumab 1.25mg/0.05mL, após o diagnóstico de Edema Macular Diabético. Devido ao edema macular persistente e resistente no olho direito após 3 injeções de Bevacizumab, intervaladas 4 semanas, foi realizado o switch terapêutico para o Aflibercept 2mg/0.05mL.\newline{}
        Resultados: Após switch terapêutico verificou-se uma boa resposta ao fármaco ao final de 3 injeções de periodicidade mensal, com tradução na acuidade visual no olho direito (20/40). Conclusões: Neste caso clínico comprovou-se a eficácia do switch terapêutico, em acordo com os dados reportados recentemente no protocolo AC e sem prejuízo para a acuidade visual final do doente. A Tomografia de Coerência Óptica permitiu guiar o tratamento de modo eficaz e atempado.
    \end{abstract_online}
    