
    \begin{abstract_online}{P4 - Biomarcadores Vasculares e de Neurodegeneração na Progressão da DMI}{%
        I. Costa$^{1}$, A. Carvalho$^{1}$, H. Andrade$^{1}$, B. Pereira$^{1,2,3}$, \underline{P. Camacho}$^{1,3,4}$}{%
        }{%
        $^1$ Escola Superior de Tecnologia da Saúde de Lisboa, Instituto Politécnico de Lisboa, Lisboa\newline{}$^2$ Instituto de Retina de Lisboa, IRL, Lisboa\newline{}$^3$  iNOVA4Health, NOVA Medical School, Faculdade de Ciências Médicas, NMS, FCM, Universidade NOVA de Lisboa\newline{}$^4$ H\&TRC-Health \& Technology Research Center, ESTeSL Escola Superior de Tecnologia da Saúde, Instituto Politécnico de Lisboa}
        Introdução:A Degenerescência Macular da Idade (DMI), considerada a principal causa de cegueira em pessoas com mais de 50 anos, (Congdon, 2004) atinge perto de 67 milhões de europeus e pode atingir um impacto económico superior aos 100 milhões de euros anuais. (Li et al., 2020) O envolvimento do fluxo da coroide, pela acumulação de lipofuscina no EPR, (Koh et al., 2017) e possível repercussão nas camadas internas da retina (Yenice et al., 2015; Zucchiatti et al., 2015) reforça a necessidade de conhecer melhor esta patologia e os possíveis biomarcadores para minimizar este problema de saúde pública.\newline{} 
        Objetivos: Quantificar e comparar as alterações de espessura do complexo das células ganglionares (CCG) e da coroide em participantes com diferentes padrões de progressão de DMI.\newline{}
        Material e Métodos: Análise retrospetiva longitudinal de participantes com idade superior a 49 anos e diagnóstico confirmado de DMI inicial/intermédia em pelo menos um olho (sem evidência de DMI avançada). Os 64 participantes selecionados através da base de dados do Instituto de Retina de Lisboa (IPL/2022/MetAllAMD\_ESTeSL) foram divididos em 4 grupos de acordo com a classificação de Roterdão para a DMI. A tomografia de coerência ótica de domínio espetral (SD-OCT) permitiu avaliar e quantificar a espessura do CCG e da coroide em dois momentos temporais (primeira visita vs. última visita) com um intervalo mínimo de 3 anos. Os dados recolhidos foram analisados através do Statistical Package for the Social Sciences (IBM SPSS 27).\newline{}
        Resultados: No anel interno do CCG, verificou-se uma espessura reduzida (p=0,001) no grupo DMI atrófica (51,3±21,4 µm) em comparação com os grupos DMI precoce (84,3±11,5 µm), DMI intermédia (77,6±16,1 µm) e DMI neovascular (88,9±16,3 µm). Na quantificação da espessura da coroide verificou-se uma redução generalizada no anel central (p=0,002) e no anel interno (p=0,001).
        \newline{}
        Conclusões: O comprometimento neurodegenerativo (CCG) ou vascular (coróide) encontrado nos grupos de DMI intermédia e atrófica podem constituir importantes indicadores de risco de progressão da doença nas fases iniciais e intermédia, mas também sobre o possível padrão de evolução para fases avançadas (atrófica ou neovascular).
    \end{abstract_online}
    