\section{XXIV Congresso Nacional de Ortoptistas - APOR}

O Congresso Nacional de Ortoptistas é organizado anualmente pela APOR - Associação Portuguesa de Ortoptistas, e segue na sua 24ª Edição. Conta com a participação de profissionais, estudantes e investigadores, e tem como objetivo a partilha de conhecimentos e experiências entre os profissionais da área da Ortóptica e ciências da visão, bem como a divulgação de trabalhos científicos e técnicos.

A APOR é uma associação profissional de direito privado, sem fins lucrativos, que tem como objetivo a promoção e desenvolvimento da profissão de Ortóptista em Portugal. Da sua missão fazem parte ainda a cooperação com outras associações e entidades nacionais e internacionais, a divulgação de informação científica e técnica, a promoção da formação e investigação científica assim como da qualidade dos cuidados de saúde prestados à população.

\section{Comissão Organizadora}
\begin{flushleft}
\begin{tabular}{lll}
Ana Maia & Carmen Oliveira &  Diana Silva \\
Dina Drogas & Diogo Marques &  Hugo Quental\\
João Ferreira & Maria Eduarda Lage & Nadine Gonçalves \\
Rodolfo Moura & Sandra Gonçalves 
\end{tabular}
\end{flushleft}

\section{Comissão Científica}
\begin{flushleft}
\begin{tabular}{l}
Ana Rita Santos, PhD, AIBILI, iCBR, ESS-IPP \\
Carla Lança, PhD, ESTeSL, CHRC ENSP-UNL \\
Catarina Mateus, PhD, T.BIO, ESS-IPP; CIBIT \\ 
Gonçalo Marques, MSc, IVLC, ESTeSL-IPL \\
Pedro Camacho, PhD, H\&TRC - ESTeSL, IPL; iNOVA4Health - FCM, UNL; IOGP \\ 
\end{tabular}
\end{flushleft}